%Simon Krenger: Lineare Algebra

\documentclass{report}
\usepackage[left=1.5in, right=1.5in, top=1.0in, bottom=1.0in]{geometry}
\usepackage{layout}
\usepackage{amsmath}
\usepackage{amsthm}
\usepackage{amssymb}
\usepackage[utf8]{inputenc}

\newtheorem{mydef}{Definition}

\title{Lineare Algebra}
\author{Simon Krenger}

\begin{document}
\maketitle
\chapter{Vektoren}
\section{Vektoralgebra}
Vektor von Punkt A zu Punkt B\begin{equation}\vec{AB} = \vec{b} - \vec{a}\end{equation}
Nullvektor\begin{equation}\vec{a} + (- \vec{a}) = \vec{0}\end{equation}
Kommutativität (+)\begin{equation}\vec{a} + \vec{b} = \vec{b} + \vec{a}\end{equation}
Assoziativität (+)\begin{equation}(\vec{a} + \vec{b}) + \vec{c} = \vec{a} + (\vec{b} + \vec{c})\end{equation}
Neutrales Element (+)\begin{equation}\vec{u} + \vec{0} = \vec{0} + \vec{u} = \vec{u}\end{equation}
Inverses Element (+)\begin{equation}\vec{u} + (- \vec{u}) = \vec{0}\end{equation}
Assozativität ($\cdot$)\begin{equation}\alpha \cdot (\beta \cdot \vec{u}) = (\alpha \cdot \beta) \cdot \vec{u}\end{equation}
Distributiv I\begin{equation}\alpha \cdot (\vec{u} + \vec{u})  = (\alpha \cdot \vec{u})+(\alpha \cdot \vec{v})\end{equation}
Betrag\begin{equation}|\vec{a}| = \sqrt{{a_x}^2+{a_y}^2+{a_z}^2}\end{equation}
Vektor mit Länge 1\begin{equation}\frac{\vec{w}}{|\vec{w}|} = \frac{1}{|\vec{w}|} \cdot \vec{w}\end{equation}
Die Vektoren $\vec{a}$, $\vec{b}$, $\vec{c}$ heissen linear unabhängig, wenn aus 
\begin{equation}\alpha \cdot \vec{a} + \beta \cdot \vec{b} + \gamma \cdot \vec{c} = \vec{0} \quad \mbox{also} \quad \alpha = \beta = \gamma = 0\end{equation}
Skalarprodukt
\begin{equation}\vec{a} \cdot \vec{b} = a_1b_1 + a_2b_2 + a_3b_3\end{equation}
\begin{equation}\vec{a} \cdot \vec{b} = |\vec{a}| \cdot |\vec{b}| \cdot \cos{\phi}\end{equation}
Der Betrag von $\vec{u}$ gleich der Quadratwurzel des Skalarprodukts von $\vec{u}$ mit sich selbst
\begin{equation}\vec{u} \cdot \vec{u} = {|\vec{u}|}^2\end{equation}
Orthogonalprojektion von $\vec{u}$ auf $\vec{a}$
\begin{equation}proj_a\vec{u} = \frac{\vec{u} \cdot \vec{a}}{{|\vec{a}|}^2} \cdot \vec{a}\end{equation}
Vektorprodukt (nur $\mathbb{R}^3$)
\begin{equation}\vec{a} \times \vec{b} = det\left(\begin{matrix}e_1 & e_2 & e_3 \\ a_1 & a_2 & a_3 \\ b_1 & b_2 & b_3\end{matrix}\right) = (a_2b_3-a_3b_2)e_1 + (a_3b_1-a_1b_3)e_2 + (a_1b_2-a_2b_1)e_3\end{equation}
\begin{equation}|\vec{a} \times \vec{b}| = |\vec{a}| \cdot |\vec{b}| \cdot \sin{\phi}\end{equation}
Spatprodukt
\begin{equation}\vec{u} \cdot (\vec{v} \times \vec{w}) = [\vec{u}, \vec{v}, \vec{w}] = \left|\begin{matrix}u_1 & u_2 & u_3 \\ v_1 & v_2 & v_3 \\ w_1 & w_2 & w_3\end{matrix}\right|\end{equation}
Volumen eines Spates
\begin{equation}V = |(\vec{a} \times \vec{b}) \cdot \vec{c}|\end{equation}
\newpage
\section{Vektorgeometrie}
\subsection{Gerade}
Vektorielle Gleichung der Geraden
\begin{equation}g: \vec{s} = \vec{p} + \lambda(\vec{q} - \vec{p}) \land \lambda \in \mathbb{R}\end{equation}
Koordinatengleichung der Geraden (Kartesische Form)
\begin{equation}Ax + By + C = 0 \quad \mbox{Normale} \quad \vec{n} = (A, B)\end{equation}
Gerade aus zwei Punkten A und B
\begin{equation}s = \vec{a} + \lambda (\vec{b} - \vec{a})\end{equation}
\subsubsection{Abstand von (0,0) von der Geraden}
\begin{enumerate}
\item Normalform berechnen ($Ax + By + C = 0$)
\item Normale bestimmen ($\vec{n} = (A, B)$)
\item Alle Koeffizienten durch Länge der Normalen dividieren
\item Koeffizient C ist nun der Vektor von $(0,0)$ zur Geraden
\item h aus C berechnen ($h =\frac{C}{\sqrt{A^2+B^2}}$)\end{enumerate}
\subsubsection{Abstand von P zu g}
\begin{equation}d(R,g) = \frac{\vec{n}}{|\vec{n}|}(r^2 - p^2)\end{equation}
\subsubsection{Abstand von P zu g ($\mathbb{R}^3$)}
$g: P, Q$ und $R \notin g$ und $\vec{q} - \vec{p} = \vec{u}$\\
\begin{equation}d(R,g) = h = \frac{|(\vec{r} \cdot \vec{p}) \times (\vec{q} - \vec{p})|}{|\vec{q} - \vec{p}|} = \frac{|(\vec{r} \cdot \vec{p}) \times \vec{u}|}{|\vec{u}|}\end{equation}
\subsubsection{Abstand zwischen zwei Geraden}
Spat zwischen den Geraden aufspannen
\begin{equation}h = \frac{|(\vec{a} \times \vec{b}) \cdot (p_1 - p_0)|}{|\vec{a} \times \vec{b}|}\end{equation}
\subsection{Ebene im Raum}
Vektorform: $\vec{s} = \vec{a} + \lambda (\vec{b} - \vec{a}) + \alpha (\vec{c} - \vec{a})$\\
Kartesische Form: $Ax+By+Cz+D=0$\\
Normale: $\vec{n} = (A, B, C) \bot \Sigma$\\\\
Bestimmung der Normalform: Zuerst $\vec{n}$ bestimmen aus $\vec{n} = (\vec{b} - \vec{a}) \times (\vec{c} - \vec{a})$, anschliessend Punkt einsetzen und $D$ ausrechnen.
\subsubsection{Abstand vom Ursprung}
\begin{equation}Ax + By + Cz + D = 0\end{equation}
\begin{equation}\frac{A}{\sqrt{A^2+B^2+C^2}}x+\frac{B}{\sqrt{A^2+B^2+C^2}}y + \frac{C}{\sqrt{A^2+B^2+C^2}}z + \frac{D}{\sqrt{A^2+B^2+C^2}} = 0\end{equation}
Damit ist
\begin{equation}h = \frac{D}{\sqrt{A^2+B^2+C^2}}\end{equation}
\subsubsection{Abstand eines Punktes zu einer Ebene}
Mit $P = (p_1, p_2, p_3)$
\begin{equation}h = d(P, \Sigma) = \frac{A \cdot p_1}{\sqrt{A^2+B^2+C^2}} + \frac{B \cdot p_2}{\sqrt{A^2+B^2+C^2}} + \frac{C \cdot p_3}{\sqrt{A^2+B^2+C^2}} + \frac{D}{\sqrt{A^2+B^2+C^2}}\end{equation}
\subsubsection{Schnittpunkt Gerade und Ebene}
$g: \vec{s} = p + \lambda \vec{u} \quad \land \quad \vec{u} = \vec{q} - \vec{p}$\\
$\Sigma: Ax+By+Cz+D=0$\\
$g: x = p_1+\lambda u_1 \quad y=p_2+\lambda u_2 \quad z = p_3 + \lambda u_3$\\\\
Anschliessend einsetzen in $\Sigma$:
\begin{equation}A(p_1+\lambda u_1) + B(p_2+\lambda u_2) + C(p_3+\lambda u_3) + D = 0\end{equation}
Das errechnete $\lambda$ dann in $\vec{s}$ einsetzen und Punkt berechnen.
\subsubsection{Reflexion Gerade an der Ebene}
$g: \vec{s} = \vec{p} + \lambda \vec{u} \quad \land \quad \vec{u} = \vec{q} - \vec{p}$\\
$\Sigma: Ax + By + Cz + D = 0$\\
Schnittpunkt $S = g \cap \Sigma$ und Spiegelpunkt $p' = \vec{p} + 2 \mu \vec{n} \quad \land \quad \vec{n} = (A,B,C)^T$
\begin{enumerate}
\item Normale $\vec{n}$ bestimmen: $\vec{n} = (A,B,C)^T$
\item Schnittpunkt Gerade / Ebene bestimmen ($\vec{s}$)
\item Anschliessend Normale von P auf $\Sigma$ bestimmen (Punkt $T: \vec{t} = \vec{p}+ \mu \vec{n}$)
\item Mit dem $\lambda$ aus der vorherigen Rechnung können wir $p'$ berechnen: $p' = \vec{p} + 2 \cdot \lambda \vec{n}$
\item Damit ist die gespiegelte Gerade: $g': \vec{s} + \alpha (\vec{p} - \vec{s})$
\end{enumerate}
\subsubsection{Winkel zwischen zwei Ebenen}
\begin{equation}\cos{\Phi} = \frac{A_1A_2+B_1B_2+C_1C_2}{\sqrt{{A_1}^2+{B_1}^2+{C_1}^2}\sqrt{{A_2}^2+{B_2}^2+{C_2}^2}}\end{equation}
\subsection{Kugel}
Gleichung der Kugel
\begin{equation}(x-u)^2+(y-v)^2=R^2\end{equation}
oder\\
Kreismittelpunkt $M$, Punkt auf Kreis $s$
\begin{equation}(\vec{s}-\vec{m}) \cdot (\vec{s} - \vec{m}) = R^2 = |\vec{s}-\vec{m}|^2\end{equation}
In $\mathbb{R}^3$
\begin{equation}(x-\frac{A}{2})^2+(y-\frac{B}{2})^2+(z-\frac{C}{2})^2=R^2\end{equation}
\subsubsection{Schnitt Gerade mit Kugel}
$g: \vec{s} = \vec{p} + \lambda \vec{u}$\\
$K: (\vec{s}-\vec{m})(\vec{s}-\vec{m}) = R^2$\\
\begin{equation}F = g \cap K: [(\vec{p} + \lambda \vec{u})-\vec{m}][(\vec{p} + \lambda \vec{u})-\vec{m}]=R^2\end{equation}
\begin{equation}\lambda^2(\vec{u} \cdot \vec{u})+2 \lambda (\vec{p}-\vec{m}) + (\vec{p}-\vec{m}) - R^2 = 0\end{equation}
\begin{eqnarray}(\vec{s}-\vec{m}) \cdot (\vec{s} - \vec{m}) & = & R^2 = |\vec{s}-\vec{m}|^2  \\
\vec{s} \cdot \vec{s} - 2 (\vec{s} \cdot \vec{m}) + \vec{m} \cdot \vec{m} & = & R^2\\
(\vec{p} + \lambda \vec{u})^2 - 2 (\vec{p} + \lambda \vec{u} \cdot \vec{m}) + \vec{m}^2 & = & R^2\end{eqnarray}
\subsection{Andere geometrische Figuren}
\subsubsection{Tetraeder}
\begin{equation}V_{Tetraeder} = \frac{1}{6}[\vec{u}, \vec{v}, \vec{w}] = \frac{1}{6}[\vec{u} \cdot (\vec{v} \times \vec{w})]\end{equation}
\chapter{Lineare Gleichungssysteme}
Koeffizientenmatrix
\begin{equation}A = \left(\begin{matrix}a_{11} & a_{12}\\a_{21} & a_{22}\end{matrix}\right)\end{equation}
Inhomogenität
\begin{equation}b = \left(\begin{matrix}b_{11} \\ b_{21}\end{matrix}\right)\end{equation}
Erweiterte Koeffizientenmatrix
\begin{equation}[A,b] = \left(\begin{matrix}a_{11} & a_{12} & b_{11}\\a_{21} & a_{22} & b_{21}\end{matrix}\right)\end{equation}
\section{Elementare Zeilenoperationen}
\begin{enumerate}\item Zeilenaustausch
\item Multiplikation mit Skalar $\neq 0$
\item Addition eines Vielfachen einer Zeile\end{enumerate}
\subsection{Gauss-Jordan Verfahren}
\begin{enumerate}\item Reduktion der erweiterten Koeffizientenmatrix
\item Lösung mit Rückwärtssubstitution\end{enumerate}
\subsection{Rang}
Wir bezeichnen: n = Anzahl Variablen\\
m = Anzahl Gleichungen\\\\
Rg(A): Rang einer Matrix\\
Rg(A) $\neq$ Rg(A|b): Keine Lösung\\
Rg(A) = Rg(A|b) $\land$ Rg(A) = n: Genau 1 Lösung\\
Rg(A) = Rg(A|b) $\land$ Rg(A) < n: Unendlich viele Lösungen\\
Mit Anzahl freie Parameter: Null(A) = n - Rg(A)
\chapter{Matrizen}
Matrix $A$ vom Typ $(m,n)$ ist ein Zahlenschema mit $m$ Zeilen und $n$ Spalten, auch $A_{m \times n}$
\begin{equation}A = \left(\begin{matrix}a_{11} & a_{12} & ... & a_{1n}\\... & ... & a_{ij} & ...\\a_{m1} & ... & ... & a_{mn}\end{matrix}\right)\end{equation}
\section{Spezielle Matrizen}
\subsection{Nullmatrix}
\begin{equation}A = \left(\begin{matrix}0 & 0 & ... & 0\\... & ... & 0 & ...\\0 & ... & ... & 0\end{matrix}\right)\end{equation}
\subsection{Einheitsmatrix}
\begin{equation}I = \left(\begin{matrix}1 & 0 & 0\\0 & 1 & 0\\0 & 0 & 1\end{matrix}\right)\end{equation}
Dabei gilt: $A \cdot I = I \cdot A = A$
\section{Matrizenoperationen}
\subsection{Addition}
Zwischen Matrizen gleichen Types
\begin{equation}A_{m \times n} \pm B_{m \times n} = C_{m \times n}\end{equation}
\subsection{Skalare Multiplikation}
\begin{equation}B = \lambda A \quad \to \quad b_{ij} = \lambda \cdot A_{ij}\end{equation}
B entspricht dem gleichen Typ wie A, Typ wird nicht geändert.
\subsection{Multiplikation}
$A_{m \times n}$ multipliziert mit $B_{p \times q}$, nur möglich falls "Anzahl Spalten von A = Anzahl Zeilen von B" ist.
\begin{equation}C_{n \times p} = A_{m \times n} \cdot B_{p \times q}\end{equation}
mit
\begin{eqnarray}c_{ij} = \vec{a}_i \cdot \vec{b}_j & \land & \vec{a}_i = \mbox{i-te Zeilenvektor von A}\\
& \land & \vec{b}_j = \mbox{j-te Spaltenvektor von B}\\
& \cdot & \mbox{Skalarprodukt analog Vektoren}\end{eqnarray}
\section{Lineare Gleichungssysteme als Matrizen}
Wenn $A\vec{x} = \vec{b}$, gesucht ist $\vec{x} = (x,y,z)^T$\\
Es gilt: $A \cdot A^{-1} = I$, damit ist $\vec{x} = A^{-1} \vec{b}$\\
D.h. bei 2x2 Matrizen:
\begin{equation}A = \left(\begin{matrix}a & b\\c & d\end{matrix}\right) \to A^{-1} = \frac{1}{ad -bc}\left(\begin{matrix}d & -b\\-c & a\end{matrix}\right)\end{equation}
\subsection{Determinanten}
\begin{equation}det(A) = |A| = ad-bc \quad \mbox{, falls} \quad A = \left(\begin{matrix}a & b\\c & d\end{matrix}\right) \end{equation}
Die Matrix A ist invertierbar, falls $det(A) \neq 0$\\
Die Determinante ist bis auf das Vorzeichen die Fläche des Parallelogramms, welches von den Vektoren $\vec{p} = (a,c)^T$ und $\vec{q} = (b,d)^T$ aufgespannt wird.
\subsection{Cramersche Regel}
Wenn
\begin{equation}A=\left(\begin{matrix}a_{11} & a_{12} & a_{13}\\a_{21} & a_{22} & a_{23}\\a_{31} & a_{32} & a_{33}\end{matrix}\right) \quad b = \left(\begin{matrix}b_1\\b_2\\b_3\end{matrix}\right)\end{equation}
ist nach der Cramerschen Regel:\\
\begin{equation}A_1=\left(\begin{matrix}b_1 & a_{12} & a_{13}\\b_2 & a_{22} & a_{23}\\b_3 & a_{32} & a_{33}\end{matrix}\right) \quad A_2=\left(\begin{matrix}a_{11} & b_1 & a_{13}\\a_{21} & b_2 & a_{23}\\a_{31} & b_3 & a_{33}\end{matrix}\right) \quad A_3=\left(\begin{matrix}a_{11} & a_{12} & b_1\\a_{21} & a_{22} & b_2\\a_{31} & a_{32} & b_3\end{matrix}\right)\end{equation}
und damit
\begin{center}
$x_1 = \frac{det(A_1)}{det(A)} \quad x_2 = \frac{det(A_2)}{det(A)} \quad x_3 = \frac{det(A_3)}{det(A)}$
\end{center}
Wann ist eine Determinante = 0?
\begin{enumerate}
\item $A$ enthält eine Zeile (oder Spalte) mit lauter 0
\item Zwei Zeilen (oder Spalten) sind gleich
\item Zwei Zeilen (oder Spalten) sind zueinander proportional
\item Eine Zeile (oder Spalte) ist eine Linearkombination der übrigen Zeilen / Spalten\end{enumerate}
\subsection{Inversion}
Eine Matrix A ist invertierbar, falls $det(A) \neq 0$. Gesucht ist die Inverse zur Matrix A:
\begin{equation}A=\left(\begin{matrix}a_{11} & a_{12} & a_{13}\\a_{21} & a_{22} & a_{23}\\a_{31} & a_{32} & a_{33}\end{matrix}\right)\end{equation}
Die Einheitsmatrix wird angehängt $(A|E)$:
\begin{equation}(A|E)=\left(\begin{matrix}a_{11} & a_{12} & a_{13} & 1 & 0 & 0\\a_{21} & a_{22} & a_{23} & 0 & 1 & 0\\a_{31} & a_{32} & a_{33} & 0 & 0 & 1\end{matrix}\right)\end{equation}
Dann nach unten mit Gauss reduzieren und nach oben mit Gauss reduzieren, bis linke Seite der Identitätsmatrix entspricht.
\chapter{Lineare Abbildungen}
\section{Streckung um Faktor A}
\begin{equation}A=\left(\begin{matrix}c & 0\\0 & c\end{matrix}\right)\end{equation}
Mit $c = 1$ ergibt sich die Identitätsmatrix
\section{Rotation}
\begin{equation}A=\left(\begin{matrix}\cos{\phi} & -\sin{\phi}\\\sin{\phi} & \cos{\phi}\end{matrix}\right) \quad A_{90}=\left(\begin{matrix}0 & -1\\1 & 0\end{matrix}\right) \end{equation}
\section{Spiegelung an Achse durch Ursprung}
\begin{equation}A=\left(\begin{matrix}\cos{2 \phi} & \sin{2 \phi} \\ \sin{2\phi} & -\cos{2\phi}\end{matrix}\right) \quad A_{x=y}=\left(\begin{matrix}0 & 1\\1 & 0\end{matrix}\right)\end{equation}
\section{Scherung}
\begin{equation}A=\left(\begin{matrix}1 & a \\ 0 & 1\end{matrix}\right)\end{equation}
\subsection{Homogene Koordinaten}
Jedem Punkt eine zusätzliche Koordinate $h$ zugewiesen. Die Transformationsmatrizen werden um eine Zeile und eine Spalte mit Einheitswerten erweitert.\\Rotation:
\begin{equation}\left(\begin{matrix}x'\\y'\\1\end{matrix}\right) = \left(\begin{matrix}\cos{\phi} & -\sin{\phi} & 0\\ \sin{\phi} & \cos{\phi} & 0 \\ 0 & 0 &  1\end{matrix}\right) \cdot \left(\begin{matrix}x\\y\\1\end{matrix}\right) \end{equation}
Skalierung:
\begin{equation}\left(\begin{matrix}x'\\y'\\1\end{matrix}\right) = \left(\begin{matrix}s_x & 0 & 0 \\ 0 & s_y & 0 \\ 0 & 0 & 1\end{matrix}\right) \cdot \left(\begin{matrix}x\\y\\1\end{matrix}\right) \end{equation}
Translation:
\begin{equation}\left(\begin{matrix}x'\\y'\\1\end{matrix}\right) = \left(\begin{matrix}1 & 0 & t_x \\ 0 & 1 & t_y \\ 0 & 0 & 1\end{matrix}\right) \cdot \left(\begin{matrix}x\\y\\1\end{matrix}\right) \end{equation}
So kann beispielsweise ein Ablauf aus "Verschiebung zum Zentrum (3, -2)", "Rotation um 90 Grad" und "Verschiebung zurück" folgendermassen beschrieben werden:
\begin{equation}\left(\begin{matrix}x'\\y'\\1\end{matrix}\right) =
\left(\begin{matrix}1 & 0 & 3 \\ 0 & 1 & -2 \\ 0 & 0 & 1\end{matrix}\right) \cdot
\left(\begin{matrix}0 & -1 & 0 \\ 1 & 0 & 0 \\ 0 & 0 & 1\end{matrix}\right) \cdot
\left(\begin{matrix}1 & 0 & -3 \\ 0 & 1 & 2 \\ 0 & 0 & 1\end{matrix}\right) \cdot\left(\begin{matrix}x\\y\\1\end{matrix}\right) \end{equation}
Die resultierende Matrix kann dann auf jeden Punkt angewendet werden (und muss nicht immer neu berechnet werden).
\chapter{Komplexe Zahlen}
\section{Definitionen}
$i$ = Imaginäre Einheit
\begin{equation}i^0 = 1\end{equation}
\begin{equation}i^1 = i\end{equation}
\begin{equation}i^2 = -1\end{equation}
\begin{equation}i^3 = i^2 \cdot i = -i\end{equation}
\begin{equation}i^4 = (i^2)^2 = 1\end{equation}
Eine komplexe Zahl setzt sich aus einem Realanteil $a$ und einem Imaginäranteil $b$ zusammen:
\begin{equation}z = a + bi\end{equation}
\subsection{Grundoperationen}
Konjugation von $z = a + bi$:
\begin{equation}\overline {z} = \overline{(a + bi)} = a-bi\end{equation}
Weiter gilt:
\begin{center}$\overline{\overline{c}} = a + bi$\\
$\overline{c_1} \pm \overline{c_2} = \overline{c_1 \pm c_2}$\\
$\overline{c_1} \cdot \overline{c_2} = \overline{c_1 \cdot c_2}$\\
$\frac{\overline{c_1}}{\overline{c_2}} = \overline{(\frac{c_1}{c_2})}$\end{center}
Betrag von $z$ ($abs()$):
\begin{equation}r = |z| = \sqrt{a^2 + b^2} = \sqrt{z \cdot \overline{z}}\end{equation}
\begin{equation}|a \cdot b| = |a||b|\end{equation}
Addition:
\begin{equation}z_1 + z_2 = (a_1 + b_1 i) + (a_2 + b_2 i) = (a_1 + a_2) + (b_1 + b_2)i\end{equation}
Differenz:
\begin{equation}z_1 - z_2 = (a_1 + b_1 i) - (a_2 + b_2 i) = (a_1 - a_2) + (b_1 - b_2)i\end{equation}
Multiplikation:
\begin{equation}z_1 \cdot z_2 = (a_1a_2 - b_1b_2) + (a_1b_2 + a_2b_1)i\end{equation}
Division:
\begin{equation}\frac{z_1}{z_2} = (\frac{a_1a_2 + b_1b_2}{{a_2}^2+{b_2}^2}) + (\frac{-a_1b_2+a_2b_1}
{{a_2}^2+{b_2}^2})i\end{equation}
\subsubsection{Spezialfälle der Grundoperationen}
\begin{equation}z  + \overline{z} = (a + bi)+(a - bi) = (a + a)+ (b -b) i = 2a\end{equation}
\begin{equation}z  - \overline{z} = (a + bi) - (a - bi) = (a - a) + (b - (-b)) i = 2ib\end{equation}
\begin{equation}z  \cdot \overline{z} = (a + bi)\cdot(a-bi) = a^2 + b^2\end{equation}
\section{Polarform}
Die Polarform einer komplexen Zahl ist definiert durch einen Winken ($\phi$) und einen Betrag $r$. Mit $z = a + bi$ ist damit:
\begin{equation}r = |z| = \sqrt{a^2 + b^2}\end{equation}
Weiter ist der Winkel ($arg()$):
\begin{equation}\tan{\phi} = \frac{b}{a}\end{equation}
Dies gilt in Quadrant I und IV. Für Quadrant II und III gilt: $\phi = \arctan{\frac{b}{a}+\pi}$.\\
Damit gilt:
\begin{equation}z = a + bi = r \cos{\phi} + i r \sin{\phi} = r \cdot (\cos{\phi} + i \sin{\phi})\end{equation}
Dies ergibt dann die Polarform:
\begin{equation}z = r (\cos{\phi} + i \sin{\phi})\end{equation}
\subsection{Operationen mit der Polarform}
Multiplikation
\begin{equation}z_1 \cdot z_2 = r_1r_2[\cos{(\phi_1 + \phi_2)} + i \sin{(\phi_1 + \phi_2)}]\end{equation}
Division
\begin{equation}\frac{z_1}{z_2} = \frac{r_1}{r_2} \cdot [\cos{(\phi_1 - \phi_2)} + i \sin{(\phi_1 - \phi_2)}]\end{equation}
Potenzen
\begin{equation}z^n = z_1 \cdot z_2 \cdot ... \cdot z_n = r^n (\cos{\phi} + i \sin{\phi})^n\end{equation}
Wurzeln
\begin{equation}z^{\frac{1}{n}} = \sqrt[n]{r}[\cos{(\frac{\phi}{n} + k\frac{2\pi}{n})} + i \sin{(\frac{\phi}{n} + k\frac{2\pi}{n})}] \quad \land \quad k = 0,1,2,...,n-1\end{equation}
\section{Exponentialschreibweise}
\begin{equation}z= r e^{i \phi}\end{equation}
Dabei stellt $r$ den Betrag und $\phi$ das Argument von $z$ dar.\\
Multiplikation
\begin{equation}c_1 \cdot c_2 = r_1 \cdot r_2 \cdot e^{i(\phi_1 + \phi_2)}\end{equation}
Division
\begin{equation}\frac{c_1}{c_2} = \frac{r_1}{r_2} e^{i(\phi_1 - \phi_2)}\end{equation}
Potenzen
\begin{equation}c^n = (r e^{i \phi})^n = r^n e^{i n \phi}\end{equation}
Wurzeln
\begin{equation}c^{\frac{1}{n}} = (r e^{i \phi})^{\frac{1}{n}} = r^{\frac{1}{n}}e^{i \frac{\phi}{n} + k \cdot \frac{360}{n}}\end{equation}
\end{document}